\section{Introduction}\label{sec:introduction}
% motivation
\QZ{need to write motivation}
%

% close related work
In general, deciding whether two given SQL queries are semantical equivalence is an undecidable problem(\QZ{cite}).
%
There has been substantially previous work has been done in this filed.
% introduce not so related work
Some of the work has been focus on to identify subset of relational algebra that proving semantical equivalence is decidable 
under set definition \QZ{cite} or under bag definition \QZ{cite}.
%
These type of work focus on the theoretical analysis of such program, which leads to few system that can be applied to real world sql queries.
%

% UW works
Another previous approach is converting SQL queries into a simple algebraic structure, 
then using a decision procedure to find the isomorphisms and homomorphisms between algebraic expressions to proving equivalence under set or bag definition.
%
There has been two system implemented based on different algebraic structure by this approach.
%
One is \textbf{COSETTE} \QZ{cite} that using \textbf{k-relations} as the algebraic structure to interpret SQL queries.
%
The other is \textbf{UDP} \QZ{cite} that using \textbf{unbounded semiring} to interpret SQL queries.
% why it is bad
While these system has been able to prove equivalence among some pair of sql queries under set or bag definition, it has one significant shortcoming.
%
These approaches do not model the semantic of many widely used features in SQL queries, such as arithmetic operation, NULL value, and filter predicates,
which limits the usage of such approach in real world queries.
%

% contribution
The first contribution of this paper is proposing to using SMT solver, which is a widely used technique in programming verification community, 
to decide the semantically equivalence of sql queries under \textbf{set} definition.
%
This approach views each SQL query as a loop-free and guaranteed terminated program. 
%
This approach allows us to use logic constrains to encode the semantic of large subset of SQL features,
including filter operation with all common used predicate, projection with complex operation, three different type join operation, union operation aggregate
operation and NULL value.
%
Because this problem is undecidable, this approach use over-approximation to be sound but incomplete.
%

The second contribution is an implemented tool \sys that can automatically decides the equivalence of sql queries under set definition.
%
Because deciding the semantical equivalence of a given pair of SQL queries is an undecidable problem, \sys is sound but incomplete.
%
Sound means if \sys gives an equivalent decision, the given pair of SQL queries are semantical equivalent.
%
Incomplete means if \sys gives an inequivalent decision, the given pair of SQL queries might be indeed semantical equivalent.
%
We apply \sys to an open benchmark set from calcite\QZ{cite}.
%
The evaluation result shows that \sys can prove semantical equivalence for more pairs of sql queries than \textbf{UDP} under set definition.
%
We also apply \sys to a large query dataset.
%
The result shows there are over 10 percent real word queries that contains at least one subquery is semantical equivalent or be contained by other queries'
subquery, which indicates a big opportunity for optimization.

% Paper outline.
The rest of this paper is organized as follows.
\autoref{sec:formalize} formally define the target SQL query and the problem.
%
\autoref{sec:overview} use a simple example to demonstrate whole system.
%
and \autoref{sec:approach} presents the \sys in details.
%
\autoref{sec:evaluation} presents an empirical evaluation of \sys.
%
\autoref{sec:related-work} compares our contributions to related work,
and
%
\autoref{sec:conclusion} concludes.

